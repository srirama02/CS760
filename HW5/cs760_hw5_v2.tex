\documentclass[a4paper]{article}
\usepackage{geometry}
\usepackage{graphicx}
\usepackage{natbib}
\usepackage{amsmath}
\usepackage{amssymb}
\usepackage{amsthm}
\usepackage{paralist}
\usepackage{epstopdf}
\usepackage{tabularx}
\usepackage{longtable}
\usepackage{multirow}
\usepackage{multicol}
\usepackage[hidelinks]{hyperref}
\usepackage{fancyvrb}
\usepackage{algorithm}
\usepackage{algorithmic}
\usepackage{float}
\usepackage{paralist}
\usepackage[svgname]{xcolor}
\usepackage{enumerate}
\usepackage{array}
\usepackage{times}
\usepackage{url}
\usepackage{fancyhdr}
\usepackage{comment}
\usepackage{environ}
\usepackage{times}
\usepackage{textcomp}
\usepackage{caption}
\usepackage{multirow}
\usepackage{bbm}

% \usepackage{kky}

\newcounter{thm}
\ifx\fact\undefined
\newtheorem{fact}[thm]{Fact}
\fi

\newcommand{\pen}{{\rm pen}}
\newcommand{\diag}{{\rm diag}}
\newcommand{\diam}{{\bf{\rm diam}}}
\newcommand{\spann}{{\bf{\rm span}}}
\newcommand{\nulll}{{\bf{\rm null}}}
% Distributions
\newcommand{\Bern}{{\bf{\rm Bern}}\,} % support of a function
\newcommand{\Categ}{{\bf{\rm Categ}}\,} % support of a function
\newcommand{\Mult}{{\bf{\rm Mult}}\,} % support of a function
\newcommand{\Dir}{{\bf{\rm Dir}}\,} % support of a function
\newcommand{\horizontalline}{\noindent\rule[0.5ex]{\linewidth}{1pt}}
\newcommand{\HRule}{\rule{\linewidth}{0.5mm}} 
\newcommand{\Hrule}{\rule{\linewidth}{0.3mm}}
\newcommand{\HRuleN}{\HRule\\} 
\newcommand{\HruleN}{\Hrule\\}
\newcommand{\superscript}[1]{{\scriptsize \ensuremath{^{\textrm{#1}}}}}
\newcommand{\supindex}[2]{#1^{(#2)}}
\newcommand{\xii}[1]{\supindex{x}{#1}}
\newcommand{\yii}[1]{\supindex{y}{#1}}
\newcommand{\zii}[1]{\supindex{z}{#1}}
\newcommand{\Xii}[1]{\supindex{X}{#1}}
\newcommand{\Yii}[1]{\supindex{Y}{#1}}
\newcommand{\Zii}[1]{\supindex{Z}{#1}}
\newcommand{\NN}{\mathbb{N}} % Natural numbers
\newcommand{\Ncal}{\mathcal{N}}
\newcommand{\Dcal}{\mathcal{D}}
\newcommand{\Lcal}{\mathcal{L}}
\newcommand{\Xcal}{\mathcal{X}}
\newcommand{\Pcal}{\mathcal{P}}
\newcommand{\Jcal}{\mathcal{J}}
\newcommand{\Rcal}{\mathcal{R}}
\newcommand{\indfone}{\mathbbm{1}}
\newcommand{\gb}{\mathbf{g}}
\newcommand{\Hb}{\mathbf{H}}
\newcommand{\Db}{\mathbf{D}}
\newcommand*{\zero}{{\bf 0}}
\newcommand*{\one}{{\bf 1}}

% Stuff mostly appearing in Statistics
\newcommand{\Xbar}{\bar{X}}
\newcommand{\Ybar}{\bar{Y}}
\newcommand{\Zbar}{\bar{Z}}
\newcommand{\Xb}{\mathbf{X}}


%%%%  brackets
\newcommand{\inner}[2]{\left\langle #1,#2 \right\rangle}
\newcommand{\rbr}[1]{\left(#1\right)}
\newcommand{\sbr}[1]{\left[#1\right]}
\newcommand{\cbr}[1]{\left\{#1\right\}}
\newcommand{\nbr}[1]{\left\|#1\right\|}
\newcommand{\abr}[1]{\left|#1\right|}

% derivatives and partial fractions
\newcommand{\differentiate}[2]{ \frac{ \ud #2}{\ud #1} }
\newcommand{\differentiateat}[3]{ \frac{ \ud #2}{\ud #1}  \Big|_{#1=#3} }
\newcommand{\partialfrac}[2]{ \frac{ \partial #2}{\partial #1} }
\newcommand{\partialfracat}[3]{ \frac{ \partial #2}{\partial #1} \Big|_{#1=#3} }
\newcommand{\partialfracorder}[3]{ \frac{ \partial^{#3} #2}{\partial^{#3} #1} }
\newcommand{\partialfracatorder}[4]{ \frac{ \partial^{#3} #2}{\partial^{#3} #1} \Big|_{#1=#4} }

\urlstyle{rm}

\setlength\parindent{0pt} % Removes all indentation from paragraphs
\theoremstyle{definition}
\newtheorem{definition}{Definition}[]
\newtheorem{conjecture}{Conjecture}[]
\newtheorem{example}{Example}[]
\newtheorem{theorem}{Theorem}[]
\newtheorem{lemma}{Lemma}
\newtheorem{proposition}{Proposition}
\newtheorem{corollary}{Corollary}


\floatname{algorithm}{Procedure}
\renewcommand{\algorithmicrequire}{\textbf{Input:}}
\renewcommand{\algorithmicensure}{\textbf{Output:}}
\newcommand{\abs}[1]{\lvert#1\rvert}
\newcommand{\norm}[1]{\lVert#1\rVert}
\newcommand{\RR}{\mathbb{R}}
\newcommand{\EE}{\mathbb{E}}
\newcommand{\PP}{\mathbb{P}}
\newcommand{\CC}{\mathbb{C}}
\newcommand{\Nat}{\mathbb{N}}
\newcommand{\br}[1]{\{#1\}}
\DeclareMathOperator*{\argmin}{arg\,min}
\DeclareMathOperator*{\argmax}{arg\,max}
\renewcommand{\qedsymbol}{$\blacksquare$}

\definecolor{dkgreen}{rgb}{0,0.6,0}
\definecolor{gray}{rgb}{0.5,0.5,0.5}
\definecolor{mauve}{rgb}{0.58,0,0.82}

\newcommand{\Var}{\mathrm{Var}}
\newcommand{\Cov}{\mathrm{Cov}}

\newcommand{\vc}[1]{\boldsymbol{#1}}
\newcommand{\xv}{\vc{x}}
\newcommand{\Sigmav}{\vc{\Sigma}}
\newcommand{\alphav}{\vc{\alpha}}
\newcommand{\muv}{\vc{\mu}}

\newcommand{\red}[1]{\textcolor{red}{#1}}

\def\x{\mathbf x}
\def\y{\mathbf y}
\def\w{\mathbf w}
\def\v{\mathbf v}
\def\E{\mathbb E}
\def\V{\mathbb V}

% TO SHOW SOLUTIONS, include following (else comment out):
\newenvironment{soln}{
    \leavevmode\color{blue}\ignorespaces
}{}

\hypersetup{
%    colorlinks,
    linkcolor={red!50!black},
    citecolor={blue!50!black},
    urlcolor={blue!80!black}
}

\geometry{
  top=1in,            % <-- you want to adjust this
  inner=1in,
  outer=1in,
  bottom=1in,
  headheight=3em,       % <-- and this
  headsep=2em,          % <-- and this
  footskip=3em,
}


\pagestyle{fancyplain}
\lhead{\fancyplain{}{Homework 5}}
\rhead{\fancyplain{}{CS 760 Machine Learning}}
\cfoot{\thepage}

\title{\textsc{Homework 5}} % Title

%%% NOTE:  Replace 'NAME HERE' etc., and delete any "\red{}" wrappers (so it won't show up as red)

\author{
  Sriram Ashokkumar\\
  908 216 3750\\
  https://github.com/srirama02/CS760/tree/main/HW5
}

\date{}

\begin{document}

\maketitle 


\textbf{Instructions:}
Use this latex file as a template to develop your homework. Submit your homework on time as a single pdf file. Please wrap your code and upload to a public GitHub repo, then attach the link below the instructions so that we can access it. Answers to the questions that are not within the pdf are not accepted. This includes external links or answers attached to the code implementation. Late submissions may not be accepted. You can choose any programming language (i.e. python, R, or MATLAB). Please check Piazza for updates about the homework. It is ok to share the experiments results and compare them with each other.

\vspace{0.1in}


\section{Clustering}

\subsection{K-means Clustering (14 points)}

\begin{enumerate}

\item \textbf{(6 Points)}
Given $n$ observations $X_1^n = \{X_1, \dots, X_n\}$, $X_i \in \Xcal$, the K-means objective
is to find $k$
($<n$) centres $\mu_1^k = \{\mu_1, \dots, \mu_k\}$, and a rule $f:\Xcal \rightarrow
\{1,\dots, K\}$ so as to minimize the objective

\begin{equation}
J(\mu_1^K, f; X_1^n) = \sum_{i=1}^n \sum_{k=1}^K \indfone(f(X_i) = k) \|X_i - \mu_k\|^2
\label{eqn:kmeans}
\end{equation}

Let $\Jcal_K(X_1^n) = \min_{\mu_1^K, f} J(\mu_1^K, f; X_1^n)$. Prove that
$\Jcal_{K}(X_1^n)$ is a non-increasing function of $K$.

\begin{soln}
  Let \( \Jcal_K(X_1^n) \) be the minimum value of the K-means objective function for a fixed \( K \), given by $
      \Jcal_K(X_1^n) = \min_{\mu_1^K, f} J(\mu_1^K, f; X_1^n)$.
  We need to show that for any \( K \) and \( K+1 \) such that \( K < K+1 < n \), $\Jcal_{K+1}(X_1^n) \leq \Jcal_K(X_1^n)$
  is true. If we have the optimal set of centers \( \mu_1^K \) that achieves the minimum \( \Jcal_K(X_1^n) \). Now if we add one
  more center(K+1 clusters), the the clustering algorithm will not have reached the minimum possible value has it hasn't
  converged yet. Based on this we know that, we can get define  a new \( f' \) which behaves exactly like \( f \) for the first \( K \) centers and does not initially assign 
  any point to the new center \( \mu_{K+1} \). This gives us $J(\mu_1^{K+1}, f'; X_1^n) = J(\mu_1^K, f; X_1^n)$. Since \( \Jcal_{K+1}(X_1^n) \) is 
  the minimum for \( K+1 \) centers, it has to be less than any other value of the objective function with \( K+1 \) 
  centers: $\Jcal_{K+1}(X_1^n) \leq J(\mu_1^{K+1}, f'; X_1^n)$. Combining the last two equations, 
  we get $\Jcal_{K+1}(X_1^n) \leq J(\mu_1^K, f; X_1^n) = \Jcal_K(X_1^n)$.

\end{soln}

\item \textbf{(8 Points)}
Consider the K-means (Lloyd's) clustering algorithm we studied in class. We
terminate the algorithm when there are no changes to the objective.
Show that the algorithm terminates in a finite number of steps.

\begin{soln}
    The algorithm K-means algorithm partitions the data space into a finite nubmer of regions. And after each interation,
    the data points are assigned to the nearest cluster center. As the dataset is finite, there is only a finte number
    of possible assignments, so there are only a finte number of ways to partition into a K clusters. 
    Each interation decreasets the objective function J until it has reached convergence, and the objective funciton is 
    bounded by zero and decreases with each iteration, so the algorith must terminate.
\end{soln}



\end{enumerate}



\subsection{Experiment (20 Points)}

In this question, we will evaluate
K-means clustering and GMM on a simple 2 dimensional problem.
First, create a two-dimensional synthetic dataset of 300 points by sampling 100 points each from the
three Gaussian distributions shown below:
\[
P_a = \Ncal\left(
\begin{bmatrix}
-1 \\ -1
\end{bmatrix},
\;
\sigma\begin{bmatrix}
2, &0.5 \\ 0.5, &1
\end{bmatrix}
\right),
\quad
P_b = \Ncal\left(
\begin{bmatrix}
1 \\ -1
\end{bmatrix},
\;
 \sigma\begin{bmatrix}
1, &-0.5 \\ -0.5, &2
\end{bmatrix}
\right),
\quad
P_c = \Ncal\left(
\begin{bmatrix}
0 \\ 1
\end{bmatrix},
\;
 \sigma\begin{bmatrix}
1 &0 \\ 0, &2
\end{bmatrix}
\right)
\]
Here, $\sigma$ is a parameter we will change to produce different datasets.\\

First implement K-means clustering and the expectation maximization algorithm for GMMs.
Execute both methods on five synthetic datasets,
generated as shown above with $\sigma \in \{0.5, 1, 2, 4, 8\}$. Finally, evaluate both methods on \emph{(i)} the clustering objective~\eqref{eqn:kmeans} and \emph{(ii)}  the clustering accuracy. For each of the two criteria, plot the value achieved by each method against $\sigma$.\\


Guidelines:
\begin{itemize} 
\item Both algorithms are only guaranteed to find only a local optimum so we recommend trying multiple
restarts and picking the one with the lowest objective value (This is~\eqref{eqn:kmeans} for K-means and the negative log likelihood for GMMs).
You may also experiment with a smart initialization
strategy (such as kmeans++).

\item
To plot the clustering accuracy,  you may treat the `label' of points generated from distribution
$P_u$ as $u$, where $u\in \{a, b, c\}$.
Assume that the cluster id $i$ returned by a method is $i\in \{1, 2, 3\}$.
Since clustering is an unsupervised learning problem, you should obtain the best possible mapping
from $\{1, 2, 3\}$ to $\{a, b, c\}$ to compute the clustering objective.
One way to do this is to compare the clustering centers returned by the method (centroids for
K-means, means for GMMs) and map them to the distribution with the closest mean.

\end{itemize}

Points break down: 7 points each for implementation of each method, 6 points for reporting of
evaluation metrics.

\begin{soln}
    \begin{figure}[h!]
        \centering
        \includegraphics[width=0.8\textwidth]{1.2.png}  
    \end{figure}
\end{soln}



\section{Linear Dimensionality Reduction}

\subsection{Principal Components Analysis  (10 points)}
\label{sec:pca}

Principal Components Analysis (PCA) is a popular method for linear dimensionality reduction. PCA attempts to find a lower dimensional subspace such that when you project the data onto the subspace as much of the information is preserved. Say we have data $X = [x_1^\top; \dots; x_n^\top] \in \RR^{n\times D}$ where  $x_i \in \RR^D$. We wish to find a $d$ ($ < D$) dimensional subspace $A = [a_1, \dots, a_d] \in \RR^{D\times d}$, such that $ a_i \in \RR^D$ and $A^\top A = I_d$, so as to maximize $\frac{1}{n} \sum_{i=1}^n \|A^\top x_i\|^2$.
\begin{enumerate}

\item  \textbf{(4 Points)}
    Suppose we wish to find the first direction $a_1$ (such that $a_1^\top a_1 = 1$) to maximize $\frac{1}{n} \sum_i (a_1^\top x_i)^2$.
    Show that $a_1$ is the first right singular vector of $X$.

    \begin{soln}
    SVD of \( X \), which can be written as \( X = U \Sigma V^{\top} \), where \( U \) and \( V \) are orthogonal matrices and \( \Sigma \) is a diagonal matrix of singular values.

    The optimization problem for finding \( a_1 \) is given by $$\max_{a_1^{\top} a_1 = 1} \frac{1}{n} \sum_{i=1}^{n} (a_1^{\top} x_i)^2$$
    This is the same as maximizing the squared projection of the data onto \( a_1 \):$$\max_{a_1^{\top} a_1 = 1} a_1^{\top} \left( \frac{1}{n} X^{\top}X \right) a_1$$

    The term \( \frac{1}{n} X^{\top}X \) is the sample covariance matrix of \( X \), denoted as \( S \). So:
    \[
    \max_{a_1^{\top} a_1 = 1} a_1^{\top} S a_1
    \]

    According to SVD the maximum is achieved when \( a_1 \) is the eigenvector of \( S \) corresponding to the largest eigenvalue. 
    As \( S \) is \( S = \frac{1}{n}X^{\top} X \), and the eigenvectors of \( X^{\top} X \) are the right singular vectors of \( X \), then \( a_1 \), which 
    maximizes the variance, must be the first right singular vector of \( X \).

\end{soln}

\item  \textbf{(6 Points)}
Given $a_1, \dots, a_k$, let $A_k = [a_1, \dots, a_k]$ and 
$\tilde{x}_i = x_i - A_kA_k^\top x_i$. We wish to find $a_{k+1}$, to maximize
$\frac{1}{n} \sum_i (a_{k+1}^\top \tilde{x}_i)^2$. Show that $a_{k+1}$ is the
$(k+1)^{th}$ right singular vector of $X$.

\begin{soln}
    
    Given \( A_k = [a_1, \dots, a_k] \), the residual is \( \tilde{x}_i = x_i - A_k A_k^\top x_i \). To find \( a_{k+1} \), we solve the following optimization problem:
    \[
    \max_{a_{k+1}^\top a_{k+1} = 1} \frac{1}{n} \sum_{i=1}^n (a_{k+1}^\top \tilde{x}_i)^2
    \]

    This gives us the eigenvector of the covariance matrix of the residuals \( \tilde{S} \) associated with the largest eigenvalue:
    \[
    \max_{a_{k+1}^\top a_{k+1} = 1} a_{k+1}^\top \tilde{S} a_{k+1}
    \]

    Since \( \tilde{S} \) is the covariance matrix of \( X \) in the space orthogonal to \( A_k \), and the SVD
    of \( X \) is \( X = U \Sigma V^\top \), the maximization of variance shows that \( a_{k+1} \) is the \((k+1)^{th}\) right 
    singular vector of \( X \).
\end{soln}


\end{enumerate}


\subsection{Dimensionality reduction via optimization (22 points)}

We will now motivate the dimensionality reduction problem from a slightly different
perspective. The resulting algorithm has many similarities to PCA.
We will refer to method as DRO.

As before, you are given data $\{x_i\}_{i=1}^n$, where $x_i \in \RR^D$. Let $X=[x_1^\top; \dots
x_n^\top] \in \RR^{n\times D}$. We suspect that the data
actually lies approximately in  a $d$ dimensional affine subspace.
Here $d<D$ and $d<n$.
Our goal, as in PCA, is to use this dataset to find a $d$ dimensional representation $z$ for each $x\in\RR^D$.
(We will assume that the span of the data has dimension larger than
$d$, but our method should work whether $n>D$ or $n<D$.)


Let $z_i\in \RR^d$ be the lower dimensional representation for $x_i$ and
let $Z = [z_1^\top; \dots; z_n^\top] \in \RR^{n\times d}$.
We wish to find parameters $A \in \RR^{D\times d}$, $b\in\RR^D$ and the lower
dimensional representation $Z\in \RR^{n\times d}$ so as to minimize 
\begin{equation}
J(A,b,Z) = \frac{1}{n} \sum_{i=1}^n \|x_i - Az_i - b\|^2 = \| X - ZA^\top - \one b^\top\|_F^2.
\label{eqn:dimobj}
\end{equation}
Here, $\|A\|^2_F = \sum_{i,j} A_{ij}^2$ is the Frobenius norm of a matrix.


\begin{enumerate}
\item \textbf{(3 Points)}
Let $M\in\RR^{d\times d}$ be an arbitrary invertible matrix and $p\in\RR^{d}$ be an arbitrary vector.
Denote, $A_2 = A_1M^{-1}$, $b_2 = b_1- A_1M^{-1}p$ and $Z_2 = Z_1 M^\top +
\one p^\top$.
Show that both
$(A_1, b_1, Z_1)$ and $(A_2, b_2, Z_2)$ achieve the same objective value $J$~\eqref{eqn:dimobj}.
\end{enumerate}

Therefore, in order to make the problem determined, we need to impose some
constraint on $Z$. We will assume that the $z_i$'s have zero mean and identity covariance.
That is,
\begin{align*}
\Zbar = \frac{1}{n} \sum_{i=1}^n z_i =\frac{1}{n} Z^\top {\bf 1}_n = 0, \hspace{0.3in} 
S = \frac{1}{n} \sum_{i=1}^n z_i z_i^\top 
= \frac{1}{n} Z^\top Z
= I_d
\end{align*}
Here, ${\bf 1}_d = [1, 1 \dots, 1]^\top \in\RR^d$ and $I_d$  is the $d\times d$ identity matrix.

\begin{soln}

    \[ J(A_1,b_1,Z_1) = \frac{1}{n} \sum_{i=1}^n \|x_i - A_1z_i - b_1\|^2_F \]
    
    Now, applying \( A_2 = A_1M^{-1} \), \( b_2 = b_1 - A_1M^{-1}p \), and \( Z_2 = Z_1 M^\top + \mathbf{1}p^\top \), the objective for \( (A_2, b_2, Z_2) \) is:
    
    \begin{align*}
    J(A_2,b_2,Z_2) &= \frac{1}{n} \sum_{i=1}^n \|x_i - A_2^\top z_{2i} - b_2^\top\|^2_F \\
    &= \frac{1}{n} \sum_{i=1}^n \|x_i - (A_1M^{-1})^\top(z_{1i}M^\top + p^\top) - (b_1 - A_1M^{-1}p)^\top\|^2_F \\
    &= \frac{1}{n} \sum_{i=1}^n \|x_i - A_1^\top z_{1i} - b_1\|^2_F \\
    &= J(A_1,b_1,Z_1)
    \end{align*}
    
\end{soln}

\begin{enumerate}
\setcounter{enumi}{1}
\item \textbf{(16 Points)}
Outline a procedure to solve the above problem. Specify how you
would obtain $A, Z, b$ which minimize the objective and satisfy the constraints.

\textbf{Hint: }The rank $k$ approximation of a matrix in Frobenius norm is obtained by
taking its SVD and then zeroing out all but the first $k$ singular values.

\begin{soln}
    We can optimize the objective function for $b$. First we need to take the gradient with respect to $b$ and set it equal to zero.\\
    
        $$\frac{\partial}{\partial b} ( \frac{1}{n} \sum_{i = 1}^{n} \|x_i - Az_i - b\|^2 ) = 0 \implies \frac{1}{n} \sum_{i = 1}^{n} -2(x_i - Az_i - b) = 0$$\\
        $$b = \frac{1}{n} \sum_{i = 1}^{n} (x_i - Az_i) \implies \frac{1}{n} \sum_{i = 1}^{n} x_i \,\;\; $$
        As we know that $\frac{1}{n} \sum_{i = 1}^{n} z_i = 0$

    
    The procedure would be: \\
    \begin{enumerate}
        \item Set the b to as the mean of X.
        \item Y = X - b
        \item SVD of $Y$ : $Y = U \Sigma V^\top$
        \item $Z = U_d$ and $A = \Sigma_dV_d^\top$; $U_d$ is the first column d columns of U and $V_d$ is the first d columns of $V$.
        \item This gives us $Y = A Z^\top$
    \end{enumerate}
\end{soln}


\item \textbf{(3 Points)}
You are given a point $x_*$ in the original $D$ dimensional space.
State the rule to obtain the $d$ dimensional
representation $z_*$ for this new point.
(If $x_*$ is some original point $x_i$ from the $D$--dimensional space, it shoud be the
$d$--dimensional representation $z_i$.)

\begin{soln}
    First we need to center the new poit by subtracting the mean \( \bar{x} \) of the original data: $x_{c*} = x_* - \bar{x}$. Then
    we need to project the centered point onto the D dimensional space using Matrix A: $z_* = A^\top x_{c*}$. A is the matrix
    that is given by the dimensionality reduction process. It contains the basis vectors for the D dimensional space.
\end{soln}


\end{enumerate}


\subsection{Experiment (34 points)}

Here we will compare the above three methods on two data sets. 

\begin{itemize}
\item We will implement three variants of PCA:
\begin{enumerate}
    \item "buggy PCA": PCA applied directly on the matrix $X$.
    \item "demeaned PCA": We subtract the mean along each dimension before applying PCA.
    \item "normalized PCA": Before applying PCA, we subtract the mean and scale each dimension so that the sample  mean and standard deviation along each dimension is $0$ and $1$ respectively.
    
\end{enumerate}



\item 
One way to study how well the low dimensional representation $Z$ captures the linear
structure in our data is to project $Z$ back to $D$ dimensions and look at the reconstruction
error. For PCA, if we mapped it to $d$ dimensions via $z = Vx$ then the
reconstruction is $V^\top z$. For the preprocessed versions, we first do this and then
reverse the preprocessing steps as well. For DRO  we just compute $Az + b$.
We will compare all methods by the reconstruction error on the datasets.

\item 
Please implement code for the methods: Buggy PCA (just take the SVD of $X$)
, Demeaned PCA,
Normalized PCA, DRO. In all cases your function should take in
an $n \times d$ data matrix and $d$ as an argument. It should return the
the $d$ dimensional representations, the estimated parameters, and the
reconstructions of these representations in $D$ dimensions. 

\item
You are given two datasets: A two Dimensional dataset with $50$ points 
\texttt{data2D.csv} and a thousand dimensional dataset with $500$ points
\texttt{data1000D.csv}. 

\item
For the $2D$ dataset use $d=1$. For the $1000D$ dataset, you need to choose
$d$. For this, observe the singular values in DRO and see if there is a clear
``knee point" in the spectrum.
Attach any figures/ Statistics you computed to justify your choice.

\begin{soln}
    I plotted the singluar values from the svd function. I found the point in the graph where there was a clear ``knee point''.
    I found this point at d=31.
    \begin{figure}[h!]
        \centering
        \includegraphics[width=0.4\textwidth]{2.3.kneePoint.png}  
    \end{figure}


\end{soln}

\item
For the $2D$ dataset you need to attach the a 
plot comparing the orignal points with the reconstructed points for all 4
methods.
For both datasets you should also report the reconstruction errors, that is the squared sum of
differences $\sum_{i=1}^n \|x_i - r(z_i)\|^2$,
where $x_i$'s are the original points and $r(z_i)$ are the $D$ dimensional points
reconstructed from the 
$d$ dimensional representation $z_i$.

\begin{soln}
    Reconstruction Errors: \\
        Reconstruction error for Buggy PCA (2D): 44.34515418673971 \\
        Reconstruction error for Demeaned PCA (2D): 0.5003042814256452 \\ 
        Reconstruction error for Normalized PCA (2D): 2.4736041727385336 \\ 
        Reconstruction error for DRO (2D): 0.5003042814256454 \\ 
        Reconstruction error for Buggy PCA (1000D): 136384.9746109073 \\ 
        Reconstruction error for Demeaned PCA (1000D): 136522.9794893014 \\ 
        Reconstruction error for Normalized PCA (1000D): 136814.29049881166 \\ 
        Reconstruction error for DRO (1000D): 136522.9794893014 \\


    \begin{figure}[h!]
        \centering
        \includegraphics[width=0.7\textwidth]{2.3.png}
        \caption{Plot comparing the original points with the reconstructed points}
    \end{figure}


\end{soln}

\item \textbf{Questions:} After you have completed the experiments, please answer the following questions.
\begin{enumerate}
\item Look at the results for Buggy PCA. The reconstruction error is bad and the
reconstructed points don't seem to well represent the original points. Why is
this? \\
\textbf{Hint: } Which subspace is Buggy PCA trying to project the points
onto?

\begin{soln}
    As the original points are used direclty in the SVD without any demeaning, the Buggy PCA is trying to project the points onto
    a subspace that is aligned with the mean of the data instead of the direction which best explains the variance in the data.
\end{soln}


\item The error criterion we are using is the average squared error 
between the original points and the reconstructed points.
In both examples DRO and demeaned PCA achieves the lowest error among all
methods. 
Is this surprising? Why?
\begin{soln}
    This is not suprising because both methods use demeaning which helps with centering the data. This helps as it makes sure that the
    pricipal components are the directions that capture the most variance in the data.
\end{soln}

\end{enumerate}

\item Point allocation:
\begin{itemize}
\item Implementation of the three PCA methods: \textbf{(6 Points)}
\item Implementation of DRO: \textbf{(6 points)}
\item Plots showing original points and reconstructed points for 2D dataset for each one of the 4 methods: \textbf{(10 points)}
\item Implementing reconstructions and reporting results for each one of the 4 methods for the 2 datasets: \textbf{(5 points)}
\item Choice of $d$ for $1000D$ dataset and appropriate justification:
\textbf{(3 Points)}
\item Questions \textbf{(4 Points)}
\end{itemize}

\end{itemize}


%\vspace{0.1in}

\vspace{0.2in}

\textbf{Answer format:}  \\
The graph bellow is in example of how a plot of one of the algorithms for the 2D dataset may look like: \\
\includegraphics[width=3in]{buggy_pca} \hspace{0.4in}
\\

The blue circles are from the original dataset and the red crosses are the reconstructed points. \\

And this is how the reconstruction error may look like for Buggy PCA for the 2D dataset: 0.886903







\bibliographystyle{apalike}
\end{document}


